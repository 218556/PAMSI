\usepackage{graphicx}
\usepackage{hyperref}
\usepackage{tikz}
\usetikzlibrary{decorations.fractals}


\newcommand{\fract}[1]{
    \begin{center}
        \tikz [
            decoration = #1
        ] \draw 
        		decorate{ decorate{ decorate{ (-8.5, 0) -- (-5.1, 0) }}}
        		decorate{ decorate{ decorate{ (-5.1, 0) -- (-1.7, 0) }}}
        		decorate{ decorate{ decorate{ (-1.7, 0) -- (1.7, 0) }}}
        		decorate{ decorate{ decorate{ (1.7, 0) -- (5.1, 0) }}}
        		decorate{ decorate{ decorate{ (5.1, 0) -- (8.5, 0) }}};
    \end{center}
}


\renewcommand{\maketitle}{\begin{titlepage}

\begin{center}
%\includegraphics[scale=1]{logo.png}
\vspace*{3cm}
\fract {Koch curve type 1}
\noindent \rule{\linewidth}{0.4mm}
\vspace*{0.5cm}

\LARGE \textsc{Raport}\\
\Large \textsc{Laboratorium 3}\\

\vspace*{0.5cm}
\large \textsc{Projektowanie Algorytmów i Metody Sztucznej Inteligencji}

\vspace*{0.5cm}
\rule{\linewidth}{0.4mm}
\vspace*{2cm}

\Large\textsc{Anna Postawka} \\ % W tym miejscu należy zamieścić 
\large\textsc{218556}\\ % listę członków zespołu projektowego

\vspace*{7cm}
%\textsc{Koło Naukowe Robotyków KoNaR }\\
%\textsc{\url{www.konar.pwr.edu.pl}}\\
\textsc{\today}\\
\end{center}
\end{titlepage}
\newpage
}